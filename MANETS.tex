\section{Ad-Hoc Networks}
An ad-hoc network is a type of wireless network that does not rely on any managed infrastructure, such as hard-wired routers. The network's nodes are responsible for determining their own routing paths and forwarding other nodes packets (i.e. acting as the routers). As explored by \cite{3YP:LORAWAN_MESH}, a single network could make use of multiple transmission mediums to reach the destination node. Route management is one of the most researched challenges when it comes to ad-hoc networks. Typically implementations fall into the proactive (table-driven) or reactive (on-demand) categories - though more scenario specific variations do exist (e.g. geographic routing). 

A \ac{manet} is a special type of ad-hoc network where nodes are expected to move, resulting in frequent changes to the network topology. If a network is sparse or operating at the limits of the transmission medium, and packet delivery is not time critical, the network can be treated as a delay-tolerant-network (\ac{dtn}). A common approach to \ac{dtn}s is to adopt store-carry-forward (\ac{scf}) behaviour; this is where intermediate nodes will keep hold of data until either a new path appears or signal strength improves \cite{3YP:DTNS}. 




\cite	{}
Route discovery, route selection, route maintenance, data forwarding, and route representation and metric. 


As the routing algorithm is for the most part independent of the data transfer mechanism, the choice of r 



undreds of variations proposed. The three considered in 




A brief description of three typical methods 

 very well researched. 




 demonstrates a \ac{lorawan} implementation that supports multiple hops using gateways. However, it



  Sacrafice three bytes on every message so all nodes know gps coordinates  
  Define three types of routing, OLSR, AODV, Geographic
  Add target id to packet