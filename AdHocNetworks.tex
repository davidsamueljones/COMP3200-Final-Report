\section{Ad-Hoc Networks}
An ad-hoc network is a type of wireless network that does not rely on any managed infrastructure, such as hard-wired routers. The network's nodes are responsible for determining their own routing paths and forwarding other nodes packets (i.e. acting as the routers). As explored by \cite{3YP:LORAWAN_MESH}, a single network could make use of multiple transmission mediums to reach the destination node. A \ac{manet} is a special type of ad-hoc network where nodes are expected to move, resulting in frequent changes to the network topology \cite{3YP:MANET_RFC2501}. If a network is sparse or operating at the limits of the transmission medium, and packet delivery is not time critical, the network can be treated as a delay-tolerant-network (\ac{dtn}). A common approach to \ac{dtn}s is to adopt store-carry-forward (\ac{scf}) behaviour; this is where intermediate nodes will keep hold of data until either a new path appears or signal strength improves \cite{3YP:DTNS}. 

Route management is the most researched challenge when it comes to ad-hoc networks \cite{3YP:MANET_RESEARCH_TRENDS} with implementations typically falling into the proactive or reactive categories - though more scenario specific variations do exist (e.g. geographic). Nodes using a proactive approach maintain a routing table for the whole network, to achieve this they rely on periodic updates from other nodes with their routing tables; these methods have low transmission delay but high ongoing overhead and adapt slowly to network changes. Nodes using a reactive approach explore the network when necessary to find a path, often by flooding route request packets; these methods have high transmission delay, but no ongoing overhead and can adapt to network changes immediately. This is the abstracted level to which routing algorithms are considered in this paper, as they are mostly independent of the data transfer mechanism. Full descriptions of many examples, including \ac{aodv} (reactive), \ac{olsr} (proactive) and \ac{lar} (geographic) can be found in \cite{3YP:ROUTING_ALGORITHMS}.