\chapter{Introduction}
Wireless control and data sharing systems implemented using a single central node to handle all communications can lead to limited deployments. Decentralisation promotes multi-hop networking, allowing flexible deployment structures over much larger areas. This paper studies the application of \ac{lora}, an emerging long range, low power, radio frequency (\ac{rf}) technology, for the decentralised use case of sparse robot swarms. It is reasoned why some conventional mobile-ad-hoc-network (\ac{manet}) methods are not appropriate to use with \ac{lora}, whilst presenting approaches tailored to the technology. Also touched upon is how prospective sparse swarm environments affect the physical radio performance of \ac{lora}.

The basis of swarm robotics is the coordination of multi-robot systems such that a common goal can be achieved. Capabilities of a swarm should exceed that of any single robot in the swarm; be that attributed to increased coverage, self-assembly methods, or varying hardware/payloads among the robots. The final factor could be due to the the target environment requiring a number of terrain traversal methods, or be a side effect of cost reduction (a robot in a swarm need not possess all sensor types). Spreading robots across large areas opens up potential for many practical applications, including terrain mapping, and search and rescue. The sophistication of robots required in these real-world scenarios can make them prohibitively expensive, which can lead to limited robot density. These are referred to as sparse swarms.
 
Unlike a centralised control approach, swarms rely on local neighbours sharing  data so that all instances can build a combined interpretation of the environment to aid decisions. This requires a point-to-point communication method, which has power usage proportionate to that of the overall robot,  i.e. power usage can exceed that of nodes in a fixed-position sensor network but not extend to long range Wi-Fi. In a real-world scenario the communications must be robust to physical environmental changes, which can stem from circumstances that are either external to the system, such as weather, or internal to the system, such as robot positioning. The latter having resulting factors such as varying separation distance or change in line of sight propagation caused by obstacles (e.g. trees and ridges). Wi-Fi signals in the 2.4GHz or 5GHz bands are easily attenuated so an alternative physical medium is required; this leads to the choice of \ac{lora}, which operates in the Sub-1GHz band.   


The main study of this paper is how \ac{lora} hardware parameters and regional regulations can be exploited to improve message throughput in single-hop scenarios. Consideration is given for how the proposed method would affect implementation of industry standard multi-hop \ac{manet} methods but the behaviour is not verified. Radio is considered using abstracted mathematical models informed by physical radio testing.


Due to the large amount of regulation in the sub-1ghz band, protocol performance is closely tied to how the regional regulations can be exploited.

  This project was structured to first study \ac{lora} theoretically to understand its advantages and limitations (Section !!!), before verifying how it actually performs in a real-world scenario (Chapter !!!).
  
  

The main classes of data sharing in this system, described further in Table !!!, are: 
\begin{enumerate}[label=\textbf{\Alph*}]
  	\item Global and local management messages for swarm control, e.g. requesting a robot of a certain type to meet at a location.\label{ref:msgTypeControl}
  	\item Broadcasts to nearby robots, e.g. found obstacles, possible routes, or full data dumps due to impending robot failure. \label{ref:msgTypeLocalBroadcast}
  	\item Scenario specific information relevant to all robots, e.g. terrain fingerprints where battery usage is higher than expected. \label{ref:msgTypeGlobalBroadcast}
  	\item Communication of a result back to a human controller, e.g. images of possible search targets.\label{ref:msgTypeUnicast}
\end{enumerate}

Class A can stick with naive approach, flooding or possible OLSR.
Classes B, C and D can all be implemented using the same mechanism for high data rate sharing 
Due to the close connection of LoRa radio and its performance the radio is studied in detail.
Focus on process of sending data in single hops, leave MANET stuff to people who know what they are doing


All sit on management band h1.4




% % Introduction to the structure of the project, researched the niches of \ac{lora}, the downsides of \ac{lora}WAN, etc...  

   
   
%
%   
%  \newpage
%  
%
%
%Class \ref{ref:msgTypeControl} messages are likely to be small, whereas the other classes will often require multiple packet messages. It would not usually be system critical for Class \ref{ref:msgTypeLocalBroadcast} and \ref{ref:msgTypeGlobalBroadcast} messages  to fail but a failed Class \label{ref:msgTypeUnicast} could result in a missed search result.
%
%
%This mesh topology is visualised and compared to a star topology in Figure !!!,
%
%
%
%This paper targets the viability and possible approaches of implementing a  using \ac{lora} as the physical medium in the case of sparse swarm robotics.
%
%
%Staying away from the constraints \ac{lora}WAN presents make AODV viable but still not preferred.
%
%information from all nodes before redistributing this and commands to all nodes.
%
%
% handling all node 
%
%
% collect data from, make decisions for, and send commands, to all other nodes. 
%
%
%dynamic topology
%Significant data transfers are required
%In the case of robot
%This is a vast research area covering topics such as data distribution algorithms, radio parameter selection 
%%
%% Packet Radio???
% 
%% What is specifically different about situation (ground, no sink, environmental factors), Rapidly changing...
%and jamming. If radio hardware is designed for a worst case scenario, the throughput for normal situations will ve greatly impaired. If the hardware is designed for an average situation. the network risks partition
%under stress because of poor radio links.
%
%
%Delay tolerant networks(DTN)
%
%The target should be to minimise airtime. Longer airtime increases collision chance and reduces throughput of the network as a whole.
%
%Separation distance is less of an issue as much existing work covers guaranteeing minimum connectivity via positioning !!!.
% 
% 
% Short data messages can be broadcast by all to get to their location
% Getting back to source can use GPS coordinates to decide whether relay is a good idea
% Local broadcasts can do a fetch respond system