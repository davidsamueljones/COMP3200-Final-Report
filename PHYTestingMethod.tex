It has been repeatedly shown that \ac{lora} transmissions can be received at distances exceeding 10km in  unobstructed environments (free-space) when antennas are highly elevated \cite{3YP:LORA_RANGE_REVIEW}. However, these ideal radio conditions are unlikely to be realistic for swarm robots operating in high-propagation environments such as forests. Therefore the first experiment in this paper attempts to identify the physical performance of \ac{lora} for the sparse swarm use case. 

\section{Methodology}
A full quantitive assessment was deemed infeasible given the sheer amount of data required to cover the full range of radio parameters and scenarios, coupled with this testing only being a project sub-goal. Therefore, a focus was taken to get enough data across a small selection of important scenarios and parameters, such that qualitative assessments could be made to aid protocol design. Due to the expected sparsity of robots, near field scenarios (when transmitter and receiver are very close) were of little interest; this left only the far field to test. For 868MHz signals this meant the distance between radios ($d_{tx}$) had to exceed 34.54cm (1 wavelength). 

 The two main transmission environments selected were free-space and in-forest; this was to give an understanding of both low-propagation and high-propagation scenarios. Data collection was mainly spread over two locations: \textbf{$L_{A}$} and \textbf{$L_{B}$} (split into \textbf{$L_{B1}$} and \textbf{$L_{B2}$}), identified in Figure \ref{fig:new_forest_map} and \ref{fig:stansted_map} respectively. All locations were rural and were therefore theoretically free from strong sources of external interference. Radio placement at each location was decided by first placing the transmitting radio (slave) at a fixed location, and then, using the furthest receivable point as the starting point for the receiving radio (master). From there the master was positioned closer towards the slave for each future test. In each scenario the main interest was ground level transmissions; however, to assess whether radio performance was actually compromised by the placement, comparative measurements were taken with an elevated antenna. 
 
  \begin{figure}[H]
    \centering
    \includegraphics[width=\textwidth]{Figures/new_forest_light.pdf}
    \caption[Test Location: The New Forest, Hampshire, UK]{
    Test positions for \textbf{$L_A$} :  The New Forest, Hampshire, UK.\\
    \ac{sp} in open with \ac{los} to other points a combination of free-space and light vegetation. Positions and pictures in Appendix \ref{sec:new_forest_test_pos}. To the left of \ac{mp}7 vegetation density increases, therefore making \ac{mp}7 the furthest position viable for close to free-space testing.
    }
    \label{fig:new_forest_map}
\end{figure}


 In terms of radio parameters, \ac{sf} was the main focus due to it being an on option mostly unique to \ac{lora}; all values were tested for this in all locations (\ac{sf} = 7, 8, 9, 10, 11, 12). Variations using the lowest (4/5) and highest (4/8) \ac{cr}s were collected to verify \ac{fec} performance in an environment with little or no burst interference. Additionally, as the maximum transmission unit is often defined by the protocol, and the target was to inform the protocol, the effects of varying packet size were taken (\ac{ps} = 20, 128, 255). The rest of the parameters were fixed. The 868.1MHz \ac{cf} was used with \ac{tp} set to 14dBm so that the collected data would be relevant in regard to the \ac{etsi} regulations. The bandwidth was fixed to 125kHz so that radio sensitivity was only affected by the \ac{sf}. The programmed \ac{pl} was set to 8 to match that used by \ac{lorawan} \cite{3YP:LORAWAN_REGIONAL_PARAMS}. The number of packets (\ac{pc}) transmitted for each configuration was set to 50; though not guaranteed, this gives reasonable expectation of a normal distribution, thus allowing typical statistical analysis to be performed. See Table \ref{tab:TestDefinitions} for full test definitions.
 
 To test the point-to-point transmissions, two identical platforms, which together could log the performance of sending and receiving \ac{lora} transmissions, were required. The platforms had to be suitable for outdoor use, be able to test multiple radio configurations whilst on location and provide a mechanism to indicate to user when the maximum range had been reached. The hardware and corresponding software created for this purpose is detailed in Section \ref{sec:testing_platform}. Although testing was split across multiple days the same dry conditions were present for each.
 
 \begin{figure}[H]
    \centering
    \begin{tabular}{c}
    \subfloat[{
    Test positions for in-forest testing (\textbf{$L_{B1}$}). \ac{sp} in forest with \ac{los} to other points continually obstructed by a combination of leaved and bare trees. Positions and pictures in Appendix \ref{sec:stansted_forest_test_pos}. Large clump of \ac{mp}s where radio reception was inconsistent.
    }]
    {\includegraphics[width=\textwidth]{Figures/stansted_forest.pdf}\label{fig:stansted_map_forest}} 
    \\
    \subfloat[{
        Test positions for free-space testing (\textbf{$L_{B2}$}). \ac{sp} in open with \ac{los} completely free-space. Positions and pictures in Appendix \ref{sec:stansted_free_test_pos}. No access to right of \ac{mp}13.
    }]{\includegraphics[width=\textwidth]{Figures/stansted_free_space.pdf}\label{fig:stansted_map_free}}
    \end{tabular}
    \caption[Test Location: Stansted Forest, West Sussex, UK]{
    Test locations for \textbf{$L_B$} :  Stansted Forest, West Sussex, UK.
    }
    \label{fig:stansted_map}
\end{figure}
