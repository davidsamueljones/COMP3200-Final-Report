\chapter{Conclusion}
This paper has presented an assessment of \ac{lora} for sparse swarm scenarios covering: physical radio performance, regulation concessions, and how that maps to prospective \ac{mac} protocol performance. Conclusions are that \ac{lora}'s physical technology is suitable with recorded transmit ranges, using typical \ac{lora} radios, far exceeding that achievable by Wi-Fi or other comparable technologies -- 1600m in free-space or 500m in high-propagation forest can be expected even with scenario compromises. 

That being said, given its naturally low-data-rate, the technology is severely handicapped by Sub-1GHz band limitations in Europe. Future work could study the more freely regulated 2.4GHz \ac{lora} implementation, though this is unlikely to be applicable to forest scenarios. The usable unicast throughput of ${\sim}15\text{KB}$ per hour (${\sim}2.5\text{KB}$ if multiple channels are required),  whilst conforming to regulation is significantly worse than can be expected from usual swarm transfer mediums. LoRa's multiple data-rate configurations could alleviate this, either through manual configuration on a deployment basis, or through a \ac{mac} protocol. 

The attempts in this paper to create said protocol were ultimately unsuccessful, with basic ALOHA and \ac{csma} approaches outperforming the proposed \ac{llbp}. A variation on \ac{llbp} with slotting or minor tweaks may be able to deliver better/equal performance, or a contention-free Mobile-LMAC \cite{3YP:WSN_BOOK} like approach may be required. The potential of new cheaper gateway devices could help deliver a solution to this problem but this is all a topic for future research.

Though overall the created simulator provided a reasonable assessment of protocols and proposed some novel modelling techniques to closely emulate real-world context, one drawback of the testing approach was that decisions were made mostly agnostic to scenario data. Given \ac{mac} layers can be very scenario specific, implementing realistic swarm data models and distribution algorithms like SOUL \cite{3YP:SOUL} would give more relatable between-protocol assessment. The protocol may need to be fully integrated into a network stack, e.g. with routing protocols, for this. Ultimately, this should lead to real-world verification with radio hardware; the proposed logging platform would be suitable for this.



%\% of intended recipients received for each message
%Reasoning for failed receive: Insufficient SNR (out of range), CRC fail (bad luck), Sync Collision/ CRC from interference
%
%Missed == This implies the receiver was busy, busy receiving at the same time or not receivable 
%No Preamble on a Wanted == This implies the SF is not adequate
%
%The protocol 
%Total helpful throughput number of bytes,  number of packets

%One approach may be to slot transmissions so there is always a good time to send announcements. 
%A more complex method could require a handshake between receivers and broadcaster using slotted schedules, but this is high overhead and massively increases nearby death
%If it costs more to send out the 

%First check band that you're going to send in
%Use knowledge of who is around from regular broadcasts on 1 band
%Send out broadcast with slots for people to reply with CTR on other band (clear to receive)
%Send out ATT (about to transmit)
%Switch to lower sf and other band

%Unable to exploit use of spreading factors as agreement must be made to change settings globally

%From PHY testing it is known that 
%Also know that if devics are moving, LOS changes in forests may suddenly cause packet failure, better to shove all data asap
 
