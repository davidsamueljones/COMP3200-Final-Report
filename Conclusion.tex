\chapter{Conclusion}


Provided a flexible research platform both physically and simulator
Demonstrated how lora features can be taken advantage of

Demonstrated how hard it is to actually make use of parameters
Future work:
Incorporation with a routing protocol
Real world testing

Incorporation with more realistic test data, have testing methodologies that carry out a combination of local broadcasts, unicasts and 
% Test in hardware
% Incorporate into full network stack

When broadcasting a packet by packet basis is better, when agreeing parameters a more structured unicast system may be required.

It is highly likely that with substantial optimisations that LLBP could be made good.
Only contention based 
Initial attempts attempted to provide slots to listeners though this didn't work
Might be that it's only efficent to unicast data with different parameters (so that it's reliable)

The infrequency of possible transmissions stop this being possible.
Mobile-LMAC is a time division approach that can adapt transmission slots dynamically, however there is no mechanism to adjust 

%A general acoustic fault detection system has been presented, together with an analysis of its individual sub-components. After a review of the techniques previ- ously used in different narrower fields of activity, a general method that aims to incorporate their main advantages was investigated. Having in mind the idea of an embedded system that can easily be trained and deployed in a broad range of en- vironments, the framework that captures the whole process, from audio recording to a final fault warning, was developed and evaluated.
%The fault detection process starts with the data gathering. A set of six different input devices was tested and although a major difference between the captured frequency ranges does not exist, improved accuracy was observed when using a 27mm, no membrane speaker attached to the exterior case of the monitored sys- tem. Following that, the audio recordings are divided into one second long seg- ments. These can be further windowed, depending on the requirements of the used feature extractor. Classification layer 1 is composed of three Hidden Markov Models and a fourth Support Vector Machine. Each of this models uses a distinct feature extractor which gives it a set of advantages and drawbacks. The goal of the project was to analyse and compare the characteristics of each model and ex- tractor and find the combination that improves the system’s overall performance and generalisation capabilities. This process resulted in a choice of three HMMs using MFCC, Spectral Rolloff and Spectral Centroid and a SVM working with a combination of Zero-Crossing Rate and Short-time Energy. On top of this, Classi- fication layer 2 consists of a HMM that observes the underlying level’s performance relative to the properties of the input, and takes a final and more reliable decision. The output label is used together with a state machine of the monitored system
% to track the running system status and raise fault warnings when a transition be- tween consecutive events is missed, or the duration of an action is abnormal. This entire process is carried out in real time, a fault warning being triggered with a maximum delay of 10 seconds.
%After the design challenges were explored, an evaluation of the framework was performed. The results show that the proposed two layered classification model has a better performance than any of the reviewed individual sub-components on the used datasets, while an increased generalisation capability was observed.
%In conclusion, as the performance and capabilities of digital systems is continually increased, new opportunities that are worth exploring emerge. It is expected that in the following years an exponential number of ”intelligent” systems will make their presence felt around us. With an increased machine complexity and a reduced number of humans working around them, new methods of fault detection and maintenance have to be developed to ensure a smooth, reliable and durable work flow.