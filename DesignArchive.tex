\chapter{Design Archive}
\begingroup
\raggedright\small
\definecolor{folderbg}{RGB}{124,166,198}
\definecolor{folderborder}{RGB}{110,144,169}
\newlength\Size
\setlength\Size{4pt}
\tikzset{%
  folder/.pic={%
    \filldraw [draw=folderborder, top color=folderbg!50, bottom color=folderbg] (-1.05*\Size,0.2\Size+5pt) rectangle ++(.75*\Size,-0.2\Size-5pt);
    \filldraw [draw=folderborder, top color=folderbg!50, bottom color=folderbg] (-1.15*\Size,-\Size) rectangle (1.15*\Size,\Size);
  },
  file/.pic={%
    \filldraw [draw=folderborder, top color=folderbg!5, bottom color=folderbg!10] (-\Size,.4*\Size+5pt) coordinate (a) |- (\Size,-1.2*\Size) coordinate (b) -- ++(0,1.6*\Size) coordinate (c) -- ++(-5pt,5pt) coordinate (d) -- cycle (d) |- (c) ;
  },
}
\forestset{%
  declare autowrapped toks={pic me}{},
  pic dir tree/.style={%
    for tree={%
      folder,
      font=\ttfamily,
      grow'=0,
    },
    before typesetting nodes={%
      for tree={%
        edge label+/.option={pic me},
      },
    },
  },
  pic me set/.code n args=2{%
    \forestset{%
      #1/.style={%
        inner xsep=2\Size,
        pic me={pic {#2}},
      }
    }
  },
  pic me set={directory}{folder},
  pic me set={file}{file},
}


Overview of folder structure in the design archive, not all low-level folders are indicated.\\
\vspace{5mm}

\begin{forest}
  pic dir tree,
  where level=0{}{
    directory,
  },
[design\_archive.zip, file
  [datalogger\_schematic
  ]
  [datalogger\_source
    [LoRa\_Datalogger
    ]
    [LoRa\_Board\_ID\_Setter
    ]
  ]
  [logged\_data
    [location\_pictures
  	]
  	[location\_results
  	]
  	[testdefs
  	]
  ]
  [processing\_scripts
  ]
  [simulator\_output\_data
  ]
  [simulator\_source
      [LoRa\_Simulator
    ]
  ]
]
\end{forest}

\textit{PlatformIO build files provided for datalogger source} \\
\textit{Maven POM (Project Object Model) file provided for building simulator.}

\textit{MATLAB scripts require the library functions \texttt{sigm\_fit}\footnote{\texttt{sigm\_fit.m}, rpavao@gmail.com, 2016}, \texttt{vline}/\texttt{hline}\footnote{\texttt{vline.m}/\texttt{hline.m}, Brandon Kuczenski, Kensington Labs, 2001} and \texttt{haversine}\footnote{\texttt{haversine.m}, Josiah Renfree, 2010}}
\endgroup



