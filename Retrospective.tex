\chapter{Retrospective}
This was very much a research-led project, and from conception, was always open to change. The most major change from the brief (Appendix \ref{sec:project_brief}) was to prioritise simulator creation over testing protocols directly on hardware. This was required as it became clear that the determined problem, managing channel access, required an infeasible number of test devices. Even if sourcing these had been possible, the protocol would need to be nearly fully defined before this stage of testing was sensible. The possibility of this was raised during the risk assessment carried out in the initial project stages. The full risk assessment, marked up with comments on risk mitigation effectiveness, is in Appendix \ref{risk_analysis}. As, initial plans only considered basic simulator logic and the change meant a more feature-full simulator with \ac{rf} modelling was required, work was re-prioritised to handle this. This is most clearly demonstrated by comparing the planned Gantt chart (Figure \ref{fig:planned_gantt_chart}) to the actual progress Gantt chart (Figure \ref{fig:actual_gantt_chart}). Unfortunately, this did lead to limited time for feeding back test results into informing the protocol design; ultimately, leading to a protocol that did not out perform simpler \ac{csma} approaches.

The project required many disciplines. Prior to this project, I had considerable experience in Java development and embedded programming but only minor experience in electronics. Whereas, principles of ad-hoc networking and \ac{rf} communications were completely new. Reference textbooks including: \cite{3YP:WSN_BOOK}, \cite{3YP:ANTENNA_BOOK} and \cite{3YP:RF_BOOK} were key to building this knowledge. Consequential skills were gained in MATLAB and statistical analysis.