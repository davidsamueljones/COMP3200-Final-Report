\chapter{Project Brief}\label{sec:project_brief}
\section*{Problem}
The premise of swarm robotics is having many robots, which are ideally relatively cheap, working together to achieve a common goal. The majority of current research targets swarms working in a small area, at high density, to solve scenarios such as obstacle traversal or close proximity positioning; these swarms have benefits available such as high data rate communications and many robots being available to solve each scenario. 

There is however a branch of research that targets swarms working over a large area; these are known as sparse swarms and have applications including mapping, and search and rescue. The robots in these swarms may be spread far apart to cover the area and should be able to work together autonomously; this means they must have a robust communication method. Ideally, the communication method should be base-station to base-station, without the need for a separate gateway. For real-world scenarios this means the communication method must cope with distance, impairments such as trees, and noise generated from the robots themselves.

\section*{Goal}
This project will study radio communications for sparse robot swarms, specifically involving the pre-mentioned real-world conditions.

\section*{Scope}
The project should define a radio communication protocol for sparse robot swarms, focusing on robustness and reliability. It should be implemented and tested in both simulations and hardware. 
