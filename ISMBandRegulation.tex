\section{\ac{ism} Band Regulation}\label{sec:ISMBandRegulation}
The industrial, scientific and medical (\ac{ism}) bands are portions of the radio spectrum, which can be used without a license, subject to local regulatory standards. These standards vary around the world but often define maximum power outputs, duty cycles (\ac{dc}s), and bandwidths. Though limits can be problematic, they help reduce the chance of internal and external system interference. Some of the most stringent regulations are in Europe, where they are controlled firstly by the \ac{etsi}\footnote{ETSI, EU, https://www.etsi.org}, and then by country specific authorities. The United States' regulations are managed by the \ac{fcc}\footnote{FCC, US, https://www.fcc.gov}, with many other countries following their example. 

Sub-1GHz \ac{lora} hardware operates around the 433MHz and 900MHz bands, but as the former is very heavily regulated for ad-hoc scenarios in the US \cite{3YP:FCC_433}, only the 900MHz area is considered viable. It should be noted that the specific available frequencies still differ between region, as is demonstrated by \ac{lorawan}'s use of the 863-870 MHz and 902-928 MHz bands for Europe and the US respectively \cite{3YP:LORAWAN_REGIONAL_PARAMS}. A comparison of the regulations can be seen in Table \ref{tab:ISMRegions}. As \ac{etsi} limits vary heavily on a by band basis, they are broken down further in Table \ref{tab:ETSIBands}.
 
\vspace{2.5mm}
\begin{table}[H]
\centering\small
\caption[900MHz regional regulation comparison]{Regional regulation comparison for 900MHz band radio \cite{3YP:FCC_900, 3YP:ETSI_HARMONISED_REG}.}
\label{tab:ISMRegions}
\renewcommand*{\arraystretch}{1.1}
\begin{tabular}{l|L{45mm}L{45mm}}
    \toprule
    & \textbf{\ac{fcc}} & \textbf{\ac{etsi}}  \\
    \midrule\addlinespace
    \textbf{Band} & 902--928MHz & 863--870MHz \\
    \textbf{\ac{eirp}} & {36dBm \newline \textit{  30dBm transmit power}} & {0.25MHz @ 27dBm \newline 6.40MHz @ $\leq$ 14dBm} \\
    \textbf{Duty Cycle} & None & 0.1\%--10\% \\
    \textbf{Bandwidth} & 26 MHz & 6.65MHz \\
    \textbf{Narrowband} & 400ms airtime per transmit & None\\
    \textbf{\ac{css}} & None & Varies  \\
    \addlinespace\bottomrule
\end{tabular}
\end{table}

\ac{dc} limits greatly reduce the amount of airtime a radio is allowed. For example, in Band h1.3, there is a 1\% \ac{dc}, which indicates a maximum of 36 seconds of airtime, over all the band's channels, over a rolling one hour period. \ac{dc}s are considered within a band so a multi-band implementation may obey multiple \ac{dc}s separately. Alternatively, the polite spectrum access (\ac{psa}) policy can be used. \ac{psa} allows airtime of up to 100 seconds per 200kHz of spectrum, per hour, regardless of \ac{dc} \cite{3YP:ETSI_PSA}. It does however require a clear channel assessment to be carried out before every transmission, independent of \ac{lora}'s \ac{cad}.

\begin{table}[H]
\centering\small
\caption[\ac{etsi} 868MHz sub-band breakdown]{\ac{etsi} 868MHz sub-bands for short range devices (adapted from \cite{3YP:ETSI_HARMONISED_REG}). Only bands relevant for \ac{css} are included. Band numbers correspond to CEPT-ERC-REC 70-03 definitions \cite{3YP:CEPT_ERC_REC}}
\label{tab:ETSIBands}
\renewcommand*{\arraystretch}{1.1}
\begin{tabular}{c|ccccc}
    \toprule
    \textbf{Band} & \textbf{Frequency} (MHz) & \textbf{\ac{eirp}} (dBm)  & \textbf{\ac{dc}} & \textbf{Max \ac{bw}} (kHz) \\
    \midrule\addlinespace
    \textbf{h1.2} & 863.00--870.00 & 14 & 0.1\% & 7000 \\
    \textbf{h1.3} & 865.00--868.00 & 14 & 1\% & 300 \\
    \textbf{h1.4} & 868.00--868.60 & 14 & 1\% & 600 \\
    \textbf{h1.5} & 868.70--869.20 & 14 & 0.1\% & 500 \\
    \textbf{h1.6} & 869.40--869.65 & 27 & 10\% & 250 \\
    \textbf{h1.7a} & 869.70--870.00 & 7 & None & 300 \\
    \textbf{h1.7b} & 869.70--870.00 & 14 & 1\% & 300 \\   
    \addlinespace\bottomrule
\end{tabular}
\end{table}

Regulations consider power as equivalent isotropically radiated power (\ac{eirp}). This value is directly related to radio transmission power and can be calculated as $\text{\ac{eirp}} = \text{\ac{tp}} - L + G$ where $L$ is cable loss and $G$ is antenna gain. The latter occurring from the antenna concentrating transmit power into a smaller. This means a radio operating in Band h1.3, using a typical omnidirectional antenna with 3dBi of gain, can only transmit at 11dBm if no cable loss occurs. Directional antenna compound these issues due to their high-gains and the low \ac{eirp} limits.

The \ac{etsi} regulations are the limiting factor in transmit power, \ac{dc} and overall available bandwidth. \ac{lora} being a \ac{css} signal means \ac{fcc} narrowband limitations need not be a concern. Under this consideration, the \ac{etsi} regulations are used as the worst-fit scenario from here on in. When considering national regulation, it is common that either all bands are implemented, or none at all \cite{3YP:CEPT_ERC_REC}. Of the ETSI bands, h1.3, h1.4 and h1.6 are of most interest. h1.3 and h1.4 giving a balanced offering of bandwidth, \ac{eirp} and \ac{dc}. Whilst h1.6 offers a far greater \ac{dc} and \ac{eirp} but limited bandwidth; a further regulatory limit means only allows a single wide-band channel can operate within this band. The maximum number of possible \ac{lora} channels in each of these bands can be seen in Table \ref{tab:ETSILoraChannels}.
\vspace{2.5mm}
\begin{table}[H]
\centering\small
\caption[Maximum channel breakdown for \ac{lora}]{Maximum channel count breakdown for relevant \ac{etsi} bands, calculated for \ac{lora}'s most common operating bandwidths with channel spacing 120\% of the \ac{bw} (to avoid inter-channel interference).}
\label{tab:ETSILoraChannels}
\renewcommand*{\arraystretch}{1.1}
\begin{tabular}{c|ccc}
    \toprule
    & \multicolumn{3}{c}{\textbf{Channel \ac{bw}}}	\\
    \textbf{Band} & \textbf{125kHz} & \textbf{250kHz} & \textbf{500kHz}\\
    \midrule\addlinespace
    \textbf{h1.3} & 19 & 8 & N/A \\
    \textbf{h1.4} & 4 & 2 & 1 \\
    \textbf{h1.6} & 1 & 1 & 0 \\
    \addlinespace\bottomrule
\end{tabular}
\end{table}
