\section{Simulator}

\subsection{Model}
The model is considered as three main components, the environment, the radios, and time. The environment is responsible for understanding node (radio) placement, the propagation channel between nodes, and all ongoing transmissions for interference/collisions. Radios create the new transmissions and poll the environment for existing transmissions. Time handles progression of the system; this is achieved using the activity-oriented paradigm. Time is considered as small sequential slices at a selectable granularity (e.g. 1ms, 5ms, 10ms). For every simulation tick, the time-slice increments and any events that have occurred or are occurring within that time slice are handled. Unlink an event-driven approach every time-slice is simulated, even if nothing new is happening. Though processor intensive, its use allows for continuous receive behaviour closer to that of a real-world scenario, aiding in the implementation of realistic radio collisions and receive behaviour.

Preamble detection looks for the 

Generic radio interface, implementations of theoretical \ac{lora} radio as well as emperical one.
Selectable free space model
Selectable model for forest model
LOS path loss model identifies objects in way and how much passes through
Events such as transmissions can be scheduled
Radios automatically receive every tick if not transmitting
Every tick radios can run custom behaviour, this allows protocols to be implemented as a listener that executes every tick
Radios have listeners for receives both failed and not
Collision, CRC Fail and Success Receive behaviour
Duty cycle managers for both lorawan method and full interval method
Metadata for protocol assessment as defined in methodology 
lora airtime
preamble sync behaviour
point interference sources
lora interference behaviour

Woodland propagation model
Propagation model is not perfect as behaviour is undefined when passing completely through a forest


to an event-driven approach,  This allows each radio to consider  protocol decisisons about partial receives, 


Each radio has its own knowledge 


imulation runs 

Granularity  Time is considered 



% Time model realtime using predicted lora airtimes
% Model discrete events

% RSSI, SNR, Path Loss

\label{sec:RadioCollisions}

% receivers are able to detect when a collision has occurred increasing liklihood of getting second packet, this has not been implemented as its performance has not been verifiable

% Sigmoid packet chance functions?
\subsection{Interface (GUI)}
