\chapter{Protocol Testing}
\section{Methodology}
Network simulations allow early assessment of basic protocol performance in controlled environments. Off-the-shelve simulation tools, such as ns-3\footnote{ns-3, https://www.nsnam.org}, offer very broad feature sets but consequently, creating an implementation with novel features, such as those present with \ac{lora} (\ac{cad}, orthogonal \ac{sf}s), is not trivial. Therefore, it was deemed more time-effective to create a specialised ad-hoc \ac{lora} simulator, using models from the \ac{phy} testing, with a subset of features relevant to the testing scenarios. The created simulator is detailed in the next section. The non-interface mode was used for gathering statistical results. Whereas, the GUI overlay was used for visually identifying node behaviour to aid understanding of statistical test results.
For each 
As the target of 
\% of intended recipients received for each message
Reasoning for failed receive: Insufficient SNR (out of range), CRC fail (bad luck), Sync Collision/ CRC from interference

Protocol 

The protocol 
Total helpful throughput number of bytes,  number of packets




Verification of the protocol is to be gauged on 
% Criteria of goodness is: 
 - \% of wanted packets received
 - \% of wanted bytes received
 - Total bytes received
 - Total packets delivered
 - Number of collisions on wanted messages

\chapter{Simulator}
\section{Model}
\subsection{Airtime}
\label{sec:LoRaTiming}
A \ac{lora} transmission consists of a number of symbols, 
Configuration of these parameters effect transmission time
The time a \ac{lora} symbol takes to transmit can be calculated



% Time model realtime using predicted lora airtimes
% Model discrete events
\subsection{Propagation}
\subsection{RSSI}
\subsection{SNR}
\subsection{Interference}
\label{sec:RadioCollisions}

% Seperate model for collisions from 91197515.pdf
% receivers are able to detect when a collision has occurred increasing liklihood of getting second packet, this has not been implemented as its performance has not been verifiable
\subsection{Receive Chance}
% Sigmoid packet chance functions?
\subsubsection{Channel Activity Detection (CAD)}
\section{Interface (GUI)}

\section{Discussion}
Approaches suited to DTN




